% VUT FIT 3BIT
% ISA 2018/2019
% Project: Programming Network Service
% Author: Vladimir Dusek, xdusek27
% Date: 30/9/2018
% File: isa-assignment.tex

%%%%%%%%%%%%%%%%%%%%%%%%%%%%%%%%%%%%%%%%%%%%%%%%%%%%%%%%%%%%%%%%%%%%%

\documentclass[11pt, a4paper, titlepage]{article}
\usepackage[left=2cm, text={17cm, 24cm}, top=3cm]{geometry}
\usepackage[utf8]{inputenc}
\usepackage[czech]{babel}
\usepackage{pdfpages}
\usepackage[obeyspaces]{url}
\usepackage{framed}
\usepackage[T1]{fontenc}
\usepackage{lmodern}
\usepackage{enumitem}

\setlength\parindent{0pt}

%%%%%%%%%%%%%%%%%%%%%%%%%%%%%%%%%%%%%%%%%%%%%%%%%%%%%%%%%%%%%%%%%%%%%

\begin{document}

\section*{ISA 2018/2019 - Projekt: Programování síťové služby}
\bigskip

%%%%%%%%%%%%%%%%%%%%%%%%%%%%%%%%%%%%%%%%%%%%%%%%%%%%%%%%%%%%%%%%%%%%%

\subsection*{Popis}

Vytvořte komunikující aplikaci podle konkrétní vybrané specifikace pomocí síťové knihovny BSD sockets (pokud není ve variantě zadání uvedeno jinak). Projekt bude vypracován v~jazyce C/C++. Pokud individuální zadání nespecifikuje vlastní referenční systém, musí být projekt přeložitelný a spustitelný na serveru merlin.fit.vutbr.cz pod operačním systémem Linux.\\

Vypracovaný projekt uložený v~archívu .tar a se jménem xlogin00.tar odevzdejte elektronicky přes IS. Soubor nekomprimujte.

%%%%%%%%%%%%%%%%%%%%%%%%%%%%%%%%%%%%%%%%%%%%%%%%%%%%%%%%%%%%%%%%%%%%%

\subsection*{Odevzdání}

\begin{itemize}
	\item Termín odevzdání je 19.11.2018 (hard deadline). Odevzdání e-mailem po uplynutí termínu, dodatečné opravy či doplnění kódu není možné.
	\item Odevzdaný projekt musí obsahovat:
	\begin{enumerate}
		\item Soubor se zdrojovým kódem (dodržujte jména souborů uvedená v~konkrétním zadání).
		\item Funkční Makefile pro překlad zdrojového souboru.
		\item Dokumentaci (soubor manual.pdf), která bude obsahovat uvedení do problematiky, návrhu aplikace, popis implementace, základní informace o~programu, návod na použití. V~dokumentaci se očekává následující: titulní strana, obsah, logické strukturování textu, přehled nastudovaných informací z~literatury, popis zajímavějších pasáží implementace, použití vytvořených programů a literatura.
		\item Soubor README obsahující krátký textový popis programu s~případnými rozšířeními/omezeními, příklad spuštění a seznam odevzdaných souborů.
		\item Další požadované soubory podle konkrétního typu zadání.
	\end{enumerate}
	\item Pokud v~projektu nestihnete implementovat všechny požadované vlastnosti, je nutné veškerá omezení jasně uvést v~dokumentaci a v~souboru README.
	\item Co není v~zadání jednoznačně uvedeno, můžete implementovat podle svého vlastního výběru. Zvolené řešení popište v~dokumentaci.
	\item Při řešení projektu respektujte zvyklosti zavedené v~OS unixového typu (jako je například formát textového souboru).
	\item Vytvořené programy by měly být použitelné a smysluplné, řádně komentované a formátované a členěné do funkcí a modulů. Program by měl obsahovat nápovědu informující uživatele o~činnosti programu a jeho parametrech. Případné chyby budou intuitivně popisovány uživateli.
	\item Aplikace nesmí v~žádném případě skončit s~chybou SEGMENTATION FAULT ani jiným násilným systémovým ukončením (např. dělení nulou).
	\item Pokud přejímáte krátké pasáže zdrojových kódů z~různých tutoriálů či příkladů z~Internetu (ne mezi sebou), tak je nutné vyznačit tyto sekce a jejich autory dle licenčních podmínek, kterými se distribuce daných zdrojových kódů řídí. V~případě nedodržení bude na projekt nahlíženo jako na plagiát.
	\item Konzultace k~projektu podává vyučující, který zadání vypsal.
	\item Před odevzdáním zkontrolujte, zda jste dodrželi všechna jména souborů požadovaná ve společné části zadání i v~zadání pro konkrétní projekt. Zkontrolujte, zda je projekt přeložitelný.
\end{itemize}


%%%%%%%%%%%%%%%%%%%%%%%%%%%%%%%%%%%%%%%%%%%%%%%%%%%%%%%%%%%%%%%%%%%%%

\subsection*{Hodnocení}

\begin{itemize}
	\item Maximální počet bodů za projekt je 20 bodů.
	\begin{itemize}
		\item Maximálně 15 bodů za plně funkční aplikace.
		\item Maximálně 5 bodů za dokumentaci. Dokumentace se hodnotí pouze v~případě funkčního kódu. Pokud kód není odevzdán nebo nefunguje podle zadání, dokumentace se nehodnotí.
	\end{itemize}
	\item Příklad kriterií pro hodnocení projektů:
	\begin{itemize}
		\item Nepřehledný, nekomentovaný zdrojový text: až -7 bodů
		\item Nefunkční či chybějící Makefile: až -4 body
		\item Nekvalitní či chybějící dokumentace: až -5 bodů
		\item Nedodržení formátu vstupu/výstupu či konfigurace: -10 body
		\item Odevzdaný soubor nelze přeložit, spustit a odzkoušet: 0 bodů
		\item Odevzdáno po termínu: 0 bodů
		\item Nedodržení zadání: 0 bodů
		\item Nefunkční kód: 0 bodů
		\item Opsáno: 0 bodů (pro všechny, kdo mají stejný kód), návrh na zahájení disciplinárního řízení.
	\end{itemize}
\end{itemize}

%%%%%%%%%%%%%%%%%%%%%%%%%%%%%%%%%%%%%%%%%%%%%%%%%%%%%%%%%%%%%%%%%%%%%

\subsection*{Varianta: Export DNS informací pomocí protokolu Syslog}

\subsubsection*{Popis}

Cílem projektu je vytvořit aplikaci, která bude umět zpracovávat data protokolu DNS (Domain Name System) a vybrané statistiky exportovat pomocí protokolu Syslog na centrální logovací server.

%%%%%%%%%%%%%%%%%%%%%%%%%%%%%%%%%%%%%%%%%%%%%%%%%%%%%%%%%%%%%%%%%%%%%

\subsubsection*{Spuštění aplikace}

\path{$ ./dns-export [-r file.pcap] [-i interface] [-s syslog-server] [-t seconds] [-h]}

\begin{itemize}
	\item \path{-r} (\path{--resource}) : zpracuje daný pcap soubor
	\item \path{-i} (\path{--interface}): naslouchá na daném síťovém rozhraní a zpracovává DNS provoz
	\item \path{-s} (\path{--server}) : hostname/ipv4/ipv6 adresa syslog serveru
	\item \path{-t} (\path{--timeout}) : doba výpočtu statistik, výchozí hodnota 60s
	\item \path{-h} (\path{--help}) : vypíše nápovědu
\end{itemize}

%%%%%%%%%%%%%%%%%%%%%%%%%%%%%%%%%%%%%%%%%%%%%%%%%%%%%%%%%%%%%%%%%%%%%

\subsubsection*{Upřesnění zadání}

Aplikace bude vytvářet následující statistiky:\\
\path{domain-name rr-type rr-answer count}\\

Pokud aplikace naslouchá na síťovém rozhraní, jsou statistiky odesílány na syslog server po vypršení definované doby dané přepínačem \path{-t}. Při zpracování \path{pcap} souboru jsou statistiky odeslány po jeho zpracování. Při obdržení signálu \path{SIGUSR1} vypíše aplikace statistiky na standardní výstup.\\

Příklad:\\
\path{google.com A~172.217.23.238 68}

%%%%%%%%%%%%%%%%%%%%%%%%%%%%%%%%%%%%%%%%%%%%%%%%%%%%%%%%%%%%%%%%%%%%%

\subsubsection*{Definice syslog zprávy}

Syslog zprávy budou dodržovat syntaxi dle RFC 5424. Povinné položky je timestamp, hostname, pri, verze, název aplikace a samotná zpráva. Lze doplnit chybějící informace, jako např. PID procesu, aj. Facility je nastaveno na local0 a severity na Informational. \\

Příklad Syslog zprávy: \\
\path{<134>1 2018-09-20T22:14:15.003Z 192.0.2.1 dns-export - - - google.com A~172.217.23.238 68} \\

Limit pro syslog zprávy je typicky 1\_kB, pro efektivnost můžete sloučit více statistických zpráv do jedné syslog zprávy. V~zjednodušené variantě lze posílat každou statistickou informaci jako samostatnou syslog zprávu.

%%%%%%%%%%%%%%%%%%%%%%%%%%%%%%%%%%%%%%%%%%%%%%%%%%%%%%%%%%%%%%%%%%%%%

\subsubsection*{Dokumentace}

Soubor Readme z~obecného zadání nahraďte souborem dns-export.1 ve formátu a syntaxi manuálové stránky - viz \url{https://liw.fi/manpages/}

Dokumentaci ve formátu pdf vytvořte dle pokynu v~obecném zadání.

%%%%%%%%%%%%%%%%%%%%%%%%%%%%%%%%%%%%%%%%%%%%%%%%%%%%%%%%%%%%%%%%%%%%%

\subsubsection*{Referenční virtuální stroj}

Implementace bude testována na standardní instalaci distribuce CentOS7. Můžete použít image pro CentOS dostupný zde - \url{http://qwe.fit.vutbr.cz/igregr/centos7.ova} (\path{user/user4lab}, \path{root/root4lab}). Alternativně lze využít server merlin, kde lze otestovat zpracování pcap souboru (pro naslouchání na síťovém rozhraní je třeba root oprávnění).

%%%%%%%%%%%%%%%%%%%%%%%%%%%%%%%%%%%%%%%%%%%%%%%%%%%%%%%%%%%%%%%%%%%%%

\end{document}
